\documentclass[12pt, letterpaper]{article}
\usepackage[letterpaper, portrait, margin=1.25in]{geometry}
\usepackage[utf8]{inputenc}
\usepackage[english]{babel}


\begin{document}

\title{%
\huge \textbf{Project Barker}\\
\large An Art Project on Social Media, Intelligent Machines, and Reality}

\author{Leo Lin, Minh Nguyen}

\maketitle

\newpage
\abstractname{ -- Project Barker is an art project that comments on the current reality of social media and human-computer interactions. Generation technology involving multiple state-of-the-art (SOTA) models helps us to build a completely "fake" Twitter. Through this interactive medium, we hope to not only trigger contemplation regarding the often underestimated power of social media and the meaning of the difference between modern social media and conventional communication technologies but also provide a space to examine humanity in front of intelligent machines and explore the boundary between the virtual world and the reality.}

\section{Introduction}

\paragraph{Social Media}Social media has become a serious societal force. It has enabled some positive change and played important roles in mass movements, such as the recent BLM movement. It also accelerates the spread of disinformation and conflict. Its potential is, so far, underestimated by the general public. Malicious entities have exploited these platforms to promote conspiracy theories, influence elections, and exacerbate genocides.

\paragraph{Artificial Intelligence}Artificial Intelligence has also profoundly impacted the world. It has transformed the medical industry, for example, with better-than-human diagnoses using computer vision and protein folding to assist drug design. However, combined with social media (in reference to recommendation algorithms), it has created echo chambers and promoted sensationalism and fake news.

\paragraph{Project Barker}Seeing that, we wanted to create a genuine yet almost satirical representation of social media and AI. Barker is a half-fledged antisocial media platform filled with AI-generated content that mimics the design of Twitter. It is not a piece of art to be marveled at, but one to experience. Exploring the app reveals information about the content’s sources and inspirations, and reading the content itself reveals not only the nature of the internet but the viewer’s very mind.

\section{Process}

\section{Conclusion}

\paragraph{}In making the app and “creating” the content, we have improved our skills in data science, programming, and UX design. More importantly, however, we learned so much about ourselves. In reading the generated content, we explored our own thoughts and easily manipulated they are. In working on the project, we gained valuable experience with working in a team in a technical project as well as insight into what we do and do not want to do in our future careers.

\end{document}